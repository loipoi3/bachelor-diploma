%!TEX root = ../abstract.tex

\abstractUkr

Кваліфікаційна робота містить: ??? стор., ??? рисунки, ??? таблиць, ??? джерел.

У даній роботі розглядаються методи для вирішення задач класифікації, а саме: MLP, який використовує оптимізаційні алгоритми в основі яких градієнтний спуск, MLP, який використовує оптимізаційний алгоритм на основі одноточкової мутації та $(1+\lambda)$-EA with GP encoding. Ці методи порівнювались в задачах бінарної та мультикласової класифікації табличних даних та картинок.

В ході дослідження було показано, що усі три методи можуть досягти однакових метрик в усіх задачах, але в контексті часу сходимості до цих метрик, найшвидшим виявився MLP, який використовує оптимізаційні алгоритми в основі яких градієнтний спуск, але в контексті контрольованості та інтерпретованості, найкраще показав себе $(1+\lambda)$-EA with GP encoding.

% наприкінці анотації потрібно зазначити ключові слова
\MakeUppercase{МАШИННЕ НАВЧАННЯ, ЕВОЛЮЦІЙНІ АЛГОРИТМИ, ГЕНЕТИЧНЕ ПРОГРАМУВАННЯ, МЕТОДИ ОПТИМІЗАЦІЇ, ЕКСПРЕСИВНІ КОДУВАННЯ}


%%%% Рішенням кафедри з 2018 року ми прибираємо анотації російською мовою
% \abstractRus
%
%Русская аннотация должна быть точным переводом украинской (включая 
%статистику и ключевые слова).

\abstractEng

This paper considers methods for solving classification problems, namely: MLP, which uses optimization algorithms based on gradient descent, MLP, which uses an optimization algorithm based on one-point mutation, and $(1+\lambda)$-EA with GP encoding. These methods were compared in the tasks of binary and multiclass classification of tabular data and pictures.

During the research, it was shown that all three methods can achieve the same metrics in all problems, but in the context of the convergence time to these metrics, MLP, which uses optimization algorithms based on gradient descent, but in the context of controllability and interpretability, showed the best performance itself $(1+\lambda)$-EA with GP encoding.

% наприкінці анотації потрібно зазначити ключові слова
\MakeUppercase{MACHINE LEARNING, EVOLUTIONARY ALGORITHMS, GENETIC PROGRAMMING, OPTIMIZATION METHODS, EXPRESSIVE ENCODINGS}

% Не прибирайте даний рядок
\clearpage