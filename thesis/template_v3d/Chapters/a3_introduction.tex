%!TEX root = ../thesis.tex
% створюємо вступ
\textbf{Актуальність дослідження.} Наразі існує дуже багато методів, які вирішують задачу класифікації: статистичні методи, методи на основі нейронних мереж, методи класичного машинного навчання. Але у цих методів є свої недоліки: методи на основі нейронних мереж, мають дуже погану інтерпретабельність та контрольованість, методи класичного машинного навчання та статистичні методи мають непогану інтерпретабельність (в залежності від методу), але також не найкращу контрольваність. Задача класифікації відіграє дуже важливу роль у сучасній науці, для прикладу, іноді потрібно класифікувати чи у людини є якась хвороба легень, маючи рентгенівський знімок легень, або класифікувати чи має людина якусь хворобу, спираючись на різні її показники, як от вага, рівень цукру в крові та інші. Контрольованість відіграє дуже важливу роль у цьому завданні, оскільки це може дозволити людині, яка запускає алгоритм надати йому якусь інформацію у якомусь вигляді, яку людина має про дані, щоб алгоритм врахував цю інформацію у своїй роботі та й загалом важливо мати можливість контролювати роботу алгоритму, оскільки це може значно покращити результат.

\textbf{Метою дослідження} порівняльний аналіз різних методів класифікації, а саме MLP з оптимізаційним алгоритмом в основі якого градієнтний спуск, MLP з оптимізаційним алгоритмом в основі якого одноточкова мутація та $(1+\lambda)$-EA with GP encodings, для підвищення контрольованості у розв'язанні задачі бінарної та мультикласової класифікації табличних даних та картинок.

\emph{Об'єктом дослідження} є якісь процеси або явища загального 
характеру (наприклад, <<інформаційні процеси в системах криптографічного 
захисту>>).

\emph{Предметом дослідження} є конкретний математичний чи фізичний 
об'єкт, який розглядається у вашій роботі та який можна трактувати
як певну властивість об'єкта дослідження (наприклад, <<моделі та методи
диференціального криптоаналізу ітеративних симетричних блочних шифрів>>).

При розв’язанні поставлених завдань використовувались такі \emph{методи дослідження}: і 
тут коротенький перелік (наприклад, але не обмежуючись: методи лінійної та абстрактної 
алгебри, теорії імовірностей, математичної статистики, комбінаторного 
аналізу, теорії кодування, теорії складності алгоритмів, методи 
комп’ютерного та статистичного моделювання) 

\textbf{Наукова новизна} отриманих результатів полягає... -- тут необхідно 
перелічити, що саме нового з точки зору науки несе ваша робота. До усіх 
тверджень, які сюди виносяться, подумки (а іноді й явним чином) потрібно 
ставити слово <<вперше>> -- і ці твердження повинні залишатись істинними.

\textbf{Практичне значення} результатів полягає... -- тут необхідно 
зазначити практичну користь від результатів вашої роботи. Що саме можна 
покращити, підвищити (або знизити), зробити гарного (або уникнути 
поганого) після вашого дослідження.

\textbf{Апробація результатів та публікації.} Наприкінці вступу необхідно 
зазначити перелік конференцій, семінарів та публікацій, в яких викладено 
результати вашої роботи. Якщо результати вашої роботи ніде не 
доповідались, опускайте даний абзац.