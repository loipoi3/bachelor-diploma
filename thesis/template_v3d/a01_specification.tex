% Титульный лист
\linespread{1.1}

\begin{center}
{\bfseries
НАЦІОНАЛЬНИЙ ТЕХНІЧНИЙ УНІВЕРСИТЕТ УКРАЇНИ \par
<<КИЇВСЬКИЙ ПОЛІТЕХНІЧНИЙ ІНСТИТУТ \par
імені Ігоря СІКОРСЬКОГО>>\par
НАВЧАЛЬНО-НАУКОВИЙ ФІЗИКО-ТЕХНІЧНИЙ ІНСТИТУТ\par
Кафедра математичного моделювання та аналізу даних}
\end{center}
\par

\linespread{1.1}
Рівень вищої освіти --- перший (бакалаврський)

Спеціальність (освітня програма) --- 113~Прикладна математика,

ОПП <<Математичні методи моделювання, розпізнавання образів та комп'ютерного зору>>

\vspace{10mm}
\begin{tabularx}{\textwidth}{XX}
& ЗАТВЕРДЖУЮ                              \\[06pt]
& В.о. завідувача кафедри                 \\[06pt]
& \rule{2.5cm}{0.25pt} І.М. Терещенко     \\[06pt]
& <<\rule{0.5cm}{0.25pt}>> \rule{2.5cm}{0.25pt} \YearOfDefence~р. 
\end{tabularx}

\vspace{5mm}
\begin{center}
{\bfseries ЗАВДАННЯ \par}
{\bfseries на дипломну роботу \par}
\end{center}

%%%%%====================================
% !!! Не чіпайте наступні три команди!
%%%%%====================================
\frenchspacing
\doublespacing          % інтервал "1,5" між рядками, тепер навічно
\setfontsize{14}

Студент: \underline{\reportAuthor} \par

1. Тема роботи: <<\emph{\reportTitle}>>,

керівник: \underline{\supervisorRegalia ~\supervisorFio}, \par
затверджені наказом по університету \No \rule{0.5cm}{0.25pt} від <<\rule{0.5cm}{0.25pt}>> \rule{2.5cm}{0.25pt} \YearOfDefence~р.

2. Термін подання студентом роботи: <<\rule{0.5cm}{0.25pt}>> \rule{2.5cm}{0.25pt} \YearOfDefence~р.

3. Вихідні дані до роботи:

4. Зміст роботи: \emph{Порівняльний аналіз багатошарового перцептрону (англ. MLP, Multilayer Perceptron) з оптимізаційними алгоритмами в основі яких градієнтний спуск, MLP з оптимізаційним алгоритмом в основі якого одноточкова мутація та $(1+\lambda)$-еволюційного алгоритму з кодуванням генетичного програмування (англ. $(1+\lambda)$-EA with GP encoding, $(1+\lambda)$-evolutionary algorithm with genetic programming encoding), на прикладі задач бінарної та мультикласової класифікації табличних даних та картинок.}

5. Перелік ілюстративного матеріалу: \emph{<<Презентація доповіді>>}

6. Дата видачі завдання: 10 грудня \YearOfBeginning~р.

% Якщо перша частина завдання вилізе за сторінку - приберіть команду \newpage
% Календарний план є продовженням завдання, а не окремою частиною

\begin{center}
	Календарний план
\end{center}

\renewcommand{\arraystretch}{1.5}
\begin{table}[h!]
	\setfontsize{14pt}
	\centering
	\begin{tabularx}{\textwidth}{|>{\centering\arraybackslash\setlength\hsize{0.25\hsize}}X|>{\setlength\hsize{2\hsize}}X|>{\centering\arraybackslash\setlength\hsize{1\hsize}}X|>{\centering\arraybackslash\setlength\hsize{0.75\hsize}}X|}
		\hline \No\par з/п                                                  & Назва етапів виконання дипломної роботи & Термін виконання & Примітка \\
		\hline
		% номер етапу
		1                                                                   &
		% назва етапу
		Узгодження теми роботи із науковим керівником                       &
		% термін виконання
		листопад-грудень \YearOfBeginning~р.                                &
		% примітка - зазвичай "Виконано"
		Виконано                                                                                                                                    \\
		%%% -- початок інтервалу для копіювання
		\hline
		% номер етапу
		2                                                                   &
		% назва етапу
		Огляд  та опрацювання опублікованих джерел за тематикою дослідження &
		% термін виконання
		грудень \YearOfBeginning~р - лютий \YearOfDefence~р.                &
		% примітка - зазвичай "Виконано"
		Виконано                                                                                                                                    \\
		\hline
		% номер етапу
		3                                                                   &
		% назва етапу
		Написання програмного забезпечення та проведення дослідження        &
		% термін виконання
		березень-квітень \YearOfDefence~р.                                  &
		% примітка - зазвичай "Виконано"
		Виконано                                                                                                                                    \\
		\hline
		% номер етапу
		4                                                                   &
		% назва етапу
		Оформлення та опис результатів                                      &
		% термін виконання
		травень \YearOfDefence~р.                                           &
		% примітка - зазвичай "Виконано"
		Виконано                                                                                                                                    \\
		\hline
		% номер етапу
		5                                                                   &
		% назва етапу
		Написання та оформлення дипломної роботи                            &
		% термін виконання
		травень-червень \YearOfDefence~р.                                   &
		% примітка - зазвичай "Виконано"
		Виконано                                                                                                                                    \\
		\hline
		% номер етапу
		6                                                                   &
		% назва етапу
		Отримання рекомендації до захисту                                   &
		% термін виконання
		08.06.2024                                                          &
		% примітка - зазвичай "Виконано"
		Виконано                                                                                                                                    \\
		%%% -- кінець інтервалу для копіювання
		%скопійовані інтервали вставляти перед фінальною \hline та заповнювати відповідно
		\hline %фінальна hline
	\end{tabularx}
\end{table}

\renewcommand{\arraystretch}{1}
\begin{tabularx}{\textwidth}{>{\setlength\hsize{1.5\hsize}}X >{\setlength\hsize{0.5\hsize}}X >{\setlength\hsize{1\hsize}}X}
	Студент  & \rule{2.5cm}{0.25pt} & \reportAuthorShort \\[06pt]
	Керівник & \rule{2.5cm}{0.25pt} & \supervisorFio     \\
\end{tabularx}

\newpage
