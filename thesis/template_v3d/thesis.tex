% ********** Приклад оформлення пояснювальної записки **********
% *********  до атестаційної роботи ступеня бакалавра **********


\documentclass{bachelor_thesis}

% Додаткові пакети вносіть у цей файл
%%%% У даний файл додавайте всі необхідні вам додаткові пакети, наприклад...


%%%% Диаграммы
%\usepackage{tikz}                      % !!! невідомий конфлікт з якимось іншим пакетом

% Пакети для кольорових текстів (необхідні для команди \todo)
%\usepackage{xcolor}                     % !!! невідомий конфлікт з якимось іншим пакетом
%\usepackage{colortbl}

\usepackage{euscript}   %ещё один красивый шрифт \EuScript
\usepackage[backend=biber, style=numeric]{biblatex}
\addbibresource{/home/loipoi/bachelor-diploma/thesis/thesis.bib}
%\usepackage[english, ukrainian]{babel}
%\usepackage{csquotes}
%%%% ...і таке інше

%\usepackage[normalem]{ulem} % для подчёркиваний uline
%\ULdepth = 0.16em % расстояние от линии до текста выше/ниже

% Додаткові визначення та перевизначення команд вносіть у цей файл
%%% У даному файлі визначайте всі необхідні вам нові команди TeX
%%% або робіть перевизначення існуючих, наприклад...

% Перевизначення символу порожньої множини та знаків "більше-дорівнює", "менше-дорівнює" на прийняті у нас
\let\oldemptyset\emptyset
\let\emptyset\varnothing
\let\geq\geqslant
\let\leq\leqslant

% Визначення нових математичних команд
\newcommand*{\binsp}[1]{\ensuremath \left\{0, 1\right\}^{#1}}       % {0, 1}^m
\newcommand*{\xor}{\ensuremath \oplus}                              % \xor = (+)
\newcommand*{\GF}[1]{\ensuremath \mathbb F_{#1}}                    % F_n
\newcommand*{\GFgroup}[1]{\ensuremath \mathbb F^{*}_{#1}}           % F^*_n
\newcommand*{\Zring}[1]{\ensuremath \mathbb Z_{#1}}                 % Z_n
\newcommand*{\Zgroup}[1]{\ensuremath \mathbb Z^{*}_{#1}}            % Z^*_n
\newcommand*{\Jset}[1]{\ensuremath \mathbb J_{#1}}                  % J_n
\newcommand*{\Qset}[1]{\ensuremath \mathbb Q_{#1}}                  % Q_n
\newcommand*{\PQset}[1]{\ensuremath \widetilde{\mathbb Q}_{#1}}     % Q~_n
\newcommand*{\cyclic}[1]{\ensuremath \left\langle {#1} \right\rangle}                  % <g>
\newcommand*{\Legendre}[2]{\ensuremath \left( \frac{#1}{#2} \right)}  % символ Лежандра/Якоби
\newcommand*{\compinv}[1]{\ensuremath {#1}^{\left\langle -1 \right\rangle}}  % обратный по композиции

% Інший спосіб визначення математичного оператору
\DeclareMathOperator{\ord}{ord}
\DeclareMathOperator{\lcm}{lcm}
\DeclareMathOperator{\Li}{Li}
\DeclareMathOperator{\Coef}{Coef}
\DeclareMathOperator{\Log}{Log}
\DeclareMathOperator{\Exp}{Exp}
\DeclareMathOperator{\Res}{Res}
\DeclareMathOperator{\charact}{char}
\DeclareMathOperator{\Sym}{Sym}


% команда для коментарів червоним кольором
% !!! Конфлікт пакету color з якимось іншим пакетом, не використовувати
%\newcommand{\todo}[1]{\textcolor{red}{#1}}


%%% ...і таке інше

% Бюрократичні відомості про автора роботи
%%% Основні відомості %%%
\newcommand{\reportAuthor}             % ПІБ автора
{Харь Дмитро Федорович}
\newcommand{\reportAuthorShort}             % ПІБ автора
{Харь Д.Ф.}
\newcommand{\reportAuthorGroup}        % група автора
{ФІ-02}
\newcommand{\reportTitle}              % Назва роботи
{Порівняння багатошарового перцептрону та $(1+\lambda)$-еволюційного алгоритму з генетичним програмуванням для задач класифікації}
%% використовуйте символ "\par" або "\\" для розбиття назви на декілька рядків

\newcommand{\supervisorFio}            % Науковий керівник, ПІБ повністю
{Яворський О.А.}
\newcommand{\supervisorRegalia}        % Науковий керівник: звання, степінь, посада
{асистент кафедри ММАД}


\newcommand{\reviewerFio}              % Рецензент, ПІБ повністю
{Прізвище І.П.}                        
\newcommand{\reviewerRegalia}          % Рецензент: звання, степінь, посада
{звання, степінь, посада}

\newcommand{\YearOfDefence}            % рік захисту
{2024}
\newcommand{\YearOfBeginning}          % попередній рік - може, можна це якось автоматизувати, нє?
{2023}


% Починаємо верстку документа
\begin{document}

\pagestyle{plain}
\setfontsize{14}

% Створюємо титульну сторінку
% Титульный лист
\thispagestyle{empty}
\linespread{1.1}

\begin{center}
{\bfseries
НАЦІОНАЛЬНИЙ ТЕХНІЧНИЙ УНІВЕРСИТЕТ УКРАЇНИ \par
<<КИЇВСЬКИЙ ПОЛІТЕХНІЧНИЙ ІНСТИТУТ \par
імені Ігоря СІКОРСЬКОГО>>\par
НАВЧАЛЬНО-НАУКОВИЙ ФІЗИКО-ТЕХНІЧНИЙ ІНСТИТУТ\par
\medskip
Кафедра математичного моделювання та аналізу даних}
\end{center}

\vspace{10mm}

\begin{tabularx}{\textwidth}{XX}
& <<До захисту допущено>> \\[06pt]
& В.о. завідувача кафедри \\[06pt]
& \rule{2.5cm}{0.25pt} І.М. Терещенко \\[06pt]
& <<\rule{0.5cm}{0.25pt}>> \rule{2.5cm}{0.25pt} \YearOfDefence~р. 
\end{tabularx}

\linespread{1.5}                    % Одинарный интервал
\begin{center}
\vspace{10mm}
{\bfseries\huge Дипломна робота \par}
{\bfseries на здобуття ступеня бакалавра \par}
\end{center}

зі спеціальності: 113 Прикладна математика \par
на тему: \textbf{<<\reportTitle>>}

\vspace{10mm}

\begin{tabularx}{\textwidth}{>{\setlength\hsize{1.5\hsize}}X >{\setlength\hsize{0.5\hsize}}X}
Виконав: студент \underline{~4~} курсу, групи \underline{\reportAuthorGroup} & \\
\underline{\reportAuthor}                                                    & \\[12pt]
Керівник: \underline{\supervisorRegalia ~\supervisorFio} & \rule{2.5cm}{0.25pt}   \\[12pt]
Рецензент: \underline{\reviewerRegalia ~\reviewerFio}    & \rule{2.5cm}{0.25pt} 
\end{tabularx}

\vspace{15mm}

\linespread{1.1}                    % Одинарный интервал
\begin{tabularx}{\textwidth}{>{\setlength\hsize{1.25\hsize}}X >{\setlength\hsize{1.5\hsize}}X >{\setlength\hsize{0.25\hsize}}X}
& Засвідчую, що у цій дипломній роботі немає запозичень з праць інших 
авторів без відповідних посилань.

& \\
& Студент \rule{2.5cm}{0.25pt}      &
\end{tabularx}

%\vspace{10mm}
\vfill
\begin{center}
{Київ~---~\YearOfDefence}
\end{center}

\newpage
\thispagestyle{plain}

% Створюємо завдання
% Титульный лист
\linespread{1.1}

\begin{center}
{\bfseries
НАЦІОНАЛЬНИЙ ТЕХНІЧНИЙ УНІВЕРСИТЕТ УКРАЇНИ \par
<<КИЇВСЬКИЙ ПОЛІТЕХНІЧНИЙ ІНСТИТУТ \par
імені Ігоря СІКОРСЬКОГО>>\par
НАВЧАЛЬНО-НАУКОВИЙ ФІЗИКО-ТЕХНІЧНИЙ ІНСТИТУТ\par
Кафедра математичного моделювання та аналізу даних}
\end{center}
\par

\linespread{1.1}
Рівень вищої освіти --- перший (бакалаврський)

Спеціальність (освітня програма) --- 113~Прикладна математика,

ОПП <<Математичні методи моделювання, розпізнавання образів та комп'ютерного зору>>

\vspace{10mm}
\begin{tabularx}{\textwidth}{XX}
& ЗАТВЕРДЖУЮ                              \\[06pt]
& В.о. завідувача кафедри                 \\[06pt]
& \rule{2.5cm}{0.25pt} Ганна ЯЙЛИМОВА     \\[06pt]
& <<\rule{0.5cm}{0.25pt}>> \rule{2.5cm}{0.25pt} \YearOfDefence~р. 
\end{tabularx}

\vspace{5mm}
\begin{center}
{\bfseries ЗАВДАННЯ \par}
{\bfseries на дипломну роботу \par}
\end{center}

%%%%%====================================
% !!! Не чіпайте наступні три команди!
%%%%%====================================
\frenchspacing
\doublespacing          % інтервал "1,5" між рядками, тепер навічно
\setfontsize{14}

Студент: \underline{\reportAuthor} \par

1. Тема роботи: <<\emph{\reportTitle}>>,

керівник: \underline{\supervisorRegalia ~\supervisorFio}, \par
затверджені наказом по університету \No \rule{0.5cm}{0.25pt} від <<\rule{0.5cm}{0.25pt}>> \rule{2.5cm}{0.25pt} \YearOfDefence~р.

2. Термін подання студентом роботи: <<\rule{0.5cm}{0.25pt}>> \rule{2.5cm}{0.25pt} \YearOfDefence~р.

3. Вихідні дані до роботи:

4. Зміст роботи: \emph{Порівняльний аналіз багатошарового перцептрону (англ. MLP, Multilayer Perceptron) з оптимізаційними алгоритмами в основі яких градієнтний спуск, MLP з оптимізаційним алгоритмом в основі якого одноточкова мутація та $(1+\lambda)$-еволюційного алгоритму з кодуванням генетичного програмування (англ. $(1+\lambda)$-EA with GP encoding, $(1+\lambda)$-evolutionary algorithm with genetic programming encoding), на прикладі задач бінарної та мультикласової класифікації табличних даних та картинок.}

5. Перелік ілюстративного матеріалу: \emph{<<Презентація доповіді>>}

6. Дата видачі завдання: 05 вересня \YearOfBeginning~р.

% Якщо перша частина завдання вилізе за сторінку - приберіть команду \newpage
% Календарний план є продовженням завдання, а не окремою частиною

\begin{center}
	Календарний план
\end{center}

\renewcommand{\arraystretch}{1.5}
\begin{table}[h!]
	\setfontsize{14pt}
	\centering
	\begin{tabularx}{\textwidth}{|>{\centering\arraybackslash\setlength\hsize{0.25\hsize}}X|>{\setlength\hsize{2\hsize}}X|>{\centering\arraybackslash\setlength\hsize{1\hsize}}X|>{\centering\arraybackslash\setlength\hsize{0.75\hsize}}X|}
		\hline \No\par з/п                                                  & Назва етапів виконання дипломної роботи & Термін виконання & Примітка \\
		\hline
		% номер етапу
		1                                                                   &
		% назва етапу
		Узгодження теми роботи із науковим керівником                       &
		% термін виконання
		листопад-грудень \YearOfBeginning~р.                                &
		% примітка - зазвичай "Виконано"
		Виконано                                                                                                                                    \\
		%%% -- початок інтервалу для копіювання
		\hline
		% номер етапу
		2                                                                   &
		% назва етапу
		Огляд  та опрацювання опублікованих джерел за тематикою дослідження &
		% термін виконання
		грудень \YearOfBeginning~р - лютий \YearOfDefence~р.                &
		% примітка - зазвичай "Виконано"
		Виконано                                                                                                                                    \\
		\hline
		% номер етапу
		3                                                                   &
		% назва етапу
		Написання програмного забезпечення та проведення дослідження        &
		% термін виконання
		березень-квітень \YearOfDefence~р.                                  &
		% примітка - зазвичай "Виконано"
		Виконано                                                                                                                                    \\
		\hline
		% номер етапу
		4                                                                   &
		% назва етапу
		Оформлення та опис результатів                                      &
		% термін виконання
		травень \YearOfDefence~р.                                           &
		% примітка - зазвичай "Виконано"
		Виконано                                                                                                                                    \\
		\hline
		% номер етапу
		5                                                                   &
		% назва етапу
		Написання та оформлення дипломної роботи                            &
		% термін виконання
		травень-червень \YearOfDefence~р.                                   &
		% примітка - зазвичай "Виконано"
		Виконано                                                                                                                                    \\
		\hline
		% номер етапу
		6                                                                   &
		% назва етапу
		Отримання рекомендації до захисту                                   &
		% термін виконання
		08.06.2024                                                          &
		% примітка - зазвичай "Виконано"
		Виконано                                                                                                                                    \\
		%%% -- кінець інтервалу для копіювання
		%скопійовані інтервали вставляти перед фінальною \hline та заповнювати відповідно
		\hline %фінальна hline
	\end{tabularx}
\end{table}

\renewcommand{\arraystretch}{1}
\begin{tabularx}{\textwidth}{>{\setlength\hsize{1.5\hsize}}X >{\setlength\hsize{0.5\hsize}}X >{\setlength\hsize{1\hsize}}X}
	Студент  & \rule{2.5cm}{0.25pt} & \reportAuthorShort \\[06pt]
	Керівник & \rule{2.5cm}{0.25pt} & \supervisorFio     \\
\end{tabularx}

\newpage


% У даному костильному рішенні перші три сторінки (титул та завдання на 
% роботу) друкуються окремо від основної частини тез.
% Тому перша сторінка сформованого документу нумерується як четверта

% Створюємо анотації
%\setcounter{page}{4}
%!TEX root = ../abstract.tex

\abstractUkr

Кваліфікаційна робота містить: 55 стор., 6 рисунки, 20 таблиць, 37 джерел.

У даній роботі розглядаються методи для вирішення задач класифікації, а саме: MLP, який використовує оптимізаційні алгоритми в основі яких градієнтний спуск, MLP, який використовує оптимізаційний алгоритм на основі одноточкової мутації та $(1+\lambda)$-EA with GP encoding. Ці методи порівнювались на задачах бінарної та мультикласової класифікації табличних даних та картинок.

У ході дослідження було встановлено, що всі три методи здатні досягти однакових метрик у всіх задачах. Найшвидшу збіжність до цих метрик продемонстрував MLP з використанням градієнтного спуску. Тим не менш, $(1+\lambda)$-EA with GP encoding виділився завдяки здатності легко адаптуватись до умов задачі. Цей метод дозволяє обирати кількість нащадків і регулювати експресивність індивідів, що надає можливість зосередити пошук рішень у конкретних областях простору рішень. Такий підхід є особливо корисним, коли потрібно зосередитися на важливих регіонах пошуку для вдосконалення рішень.


% наприкінці анотації потрібно зазначити ключові слова
\MakeUppercase{МАШИННЕ НАВЧАННЯ, ЕВОЛЮЦІЙНІ АЛГОРИТМИ, ГЕНЕТИЧНЕ ПРОГРАМУВАННЯ, МЕТОДИ ОПТИМІЗАЦІЇ, ЕКСПРЕСИВНІ КОДУВАННЯ}


%%%% Рішенням кафедри з 2018 року ми прибираємо анотації російською мовою
% \abstractRus
%
%Русская аннотация должна быть точным переводом украинской (включая 
%статистику и ключевые слова).

\abstractEng

This paper considers methods for solving classification problems, namely: MLP, which uses optimization algorithms based on gradient descent, MLP, which uses an optimization algorithm based on one-point mutation, and $(1+\lambda)$-EA with GP encoding. These methods were compared in the tasks of binary and multiclass classification of tabular data and pictures.

During the research, it was established that all three methods are able to achieve the same metrics in all tasks. The fastest convergence to these metrics was demonstrated by MLP using gradient descent. Nevertheless, $(1+\lambda)$-EA with GP encoding stood out due to its ability to easily adapt to the task conditions. This method allows you to choose the number of offsprings and regulate the expressiveness of individuals, which makes it possible to focus the search for solutions in specific areas of the solution space. This approach is particularly useful when focusing on important search regions to improve solutions.

% наприкінці анотації потрібно зазначити ключові слова
\MakeUppercase{MACHINE LEARNING, EVOLUTIONARY ALGORITHMS, GENETIC PROGRAMMING, OPTIMIZATION METHODS, EXPRESSIVE ENCODINGS}

% Не прибирайте даний рядок
\clearpage

% Створюємо зміст
%\pagenumbering{gobble}
\tableofcontents
\cleardoublepage
%\pagenumbering{arabic}
%\setcounter{page}{8}    %!!! -- продумати, як автоматизувати номер сторінки

% Створюємо перелік умовних позначень, скорочень і термінів
% Якщо цей розділ вам не потрібен, просто закоментуйте два наступних рядка
\shortings
%!TEX root = ../thesis.tex
% створюємо перелік умовних позначень, скорочень і термінів
ML --- машинне навчання (англ. Machine Learning)

MLP --- багатошаровий перцептрон (англ. Multilayer Perceptron)

EA --- еволюційний алгоритм (англ. Evolutionary Algorithm)

GP --- генетичне програмування (англ. Genetic Programming)

Adam --- адаптивна оцінка моменту (англ. Adaptive Moment Estimation)

$(1+\lambda)$-EA with GP encodings --- еволюційний алгоритм, який може використовуватися для вирішення задач класифікації (англ. $(1+\lambda)$-Evolutionary Algorithm with Genetic Programming encodings).

MLP with gradient descent --- багатошаровий перцептрон, який використовує метод на основі градієнтного спуску, в якості оптимізаційного алгоритму.

MLP with single-point mutation --- багатошаровий перцептрон, який використовує одноточкову мутацію, в якості оптимізаційного алгоритму.

Фітнес-функція --- це функція \( F: \mathcal{S} \rightarrow \mathbb{R} \), яка відображає представлення рішення \( S \) на дійсне число \( f \).

Індивід --- \( I \) в еволюційних алгоритмах визначається як кортеж \( I = (S, f) \), де \( S \) є представленням рішення в просторі рішень \( \mathcal{S} \), природа \( S \) залежить від конкретної проблеми та може варіюватися в широких межах, від двійкових рядків, дійсних векторів, дерев до більш складних структур даних, \( f \) — це значення фітнес-функції, пов’язане з індивідом, кількісно оцінюючи якість індивіда як рішення цільової проблеми.

Популяція --- \( P \) визначається як множина індивідів \( P = \{I_1, I_2, \ldots, I_N\} \), де кожен окремий \( I_i \) є варіантом вирішення розв'язуваної проблеми.

Кросинговер --- \( C \), є бінарною функцією, яка бере два індивіди з популяції як вхідні дані та створює одне або більше нащадків, потенційно включаючи генетичний матеріал від обох батьків. Формально це можна виразити так: \( C: (I_i, I_j) \rightarrow (I_{i'}, I_{j'}) \), де \( I_i \) і \( I_j \) є батьківськими індивідами, кожен з яких містить представлення рішення та значення фітнес-функції, \( I_{i'} \) і \( I_{j'} \) є особинами-нащадками, отриманими в результаті операції кросинговеру.

Мутація --- це функція \( M: I \rightarrow I' \), де \( I \) — оригінальний індивід, \( I' \) є мутованою особиною з потенційно зміненим представленням рішення \( S' \) і відповідним новим значенням фітнес-функції \( f' \), яка застосовує стохастичну модифікацію до індивіду, що потенційно призводить до появи нового варіанту рішення.

Контрольованість у контексті $(1+\lambda)$-еволюційного алгоритму з генетичним програмуванням --- визначається як здатність алгоритму дозволяти користувачу точно регулювати його параметри (наприклад, кількість нащадків $\lambda$ і стратегії мутації), щоб оптимізувати процес пошуку рішення та адаптувати його під специфічні умови задачі.

Precision --- це метрика, яка визначає відношення кількості правильно класифікованих позитивних прикладів до загальної кількості прикладів, що були класифіковані як позитивні.

Recall --- це метрика, яка визначає відношення кількості правильно класифікованих позитивних прикладів до загальної кількості справді позитивних прикладів.

F1-score --- це гармонійне середнє між precision та recall.

% Створюємо вступ
\intro
%!TEX root = ../thesis.tex
% створюємо вступ
\textbf{Актуальність дослідження.} Наразі існує дуже багато методів, які вирішують задачу класифікації: статистичні методи, методи на основі нейронних мереж, методи класичного машинного навчання. Але у цих методів є свої недоліки: методи на основі нейронних мереж, мають дуже погану інтерпретабельність та контрольованість, методи класичного машинного навчання та статистичні методи мають непогану інтерпретабельність (в залежності від методу), але також не найкращу контрольваність. Задача класифікації відіграє дуже важливу роль у сучасній науці, для прикладу, іноді потрібно класифікувати чи у людини є якась хвороба легень, маючи рентгенівський знімок легень, або класифікувати чи має людина якусь хворобу, спираючись на різні її показники, як от вага, рівень цукру в крові та інші. Можливість контролювати роботу алгоритму відіграє в цьому завданні важливу роль, оскільки це дозволяє нам налаштувати алгоритм під наші потреби і при цьому ми будемо точно знати чому та як він працює.

\textbf{Метою дослідження} є пошук найкращого методу класифікації, на прикладі методів MLP with gradient descent, MLP with single-point mutation, $(1+\lambda)$-EA with GP encodings, для підвищення продуктивності у розв'язанні задач бінарної та мультикласової класифікації табличних даних та картинок.

\emph{Об'єктом дослідження} є методи класифікації даних для машинного навчання на прикладі задачі бінарної та мультикласової класифікації.

\emph{Предметом дослідження} є особливості контролювання алгоритмів на прикладі MLP with gradient descent, MLP with single-point mutation, $(1+\lambda)$-EA with GP encodings на прикладі застосування до задач бінарної та мультикласової класифікації табличних даних та картинок.

\textbf{Наукова новизна} полягає в дослідженні та порівнянні алгоритмів MLP with gradient descent, MLP with single-point mutation, $(1+\lambda)$-EA with GP encodings на прикладі задач бінарної та мультикласової класифікації.

\textbf{Практичне значення} результатів полягає в використанні перелічених вище методів, для задачі класифікації, для покращення контрольованості і збереженню такої ж точності та швидкості, як і в класичних методах.

% Додаємо глави
% Якщо ваша робота містить менше або більше глав - модифікуйте наступні 
% рядки відповідним чином
%!TEX root = ../thesis.tex

\chapter{Методи та підходи вирішення задач класифікації}
\label{chap:review}  %% відмічайте кожен розділ певною міткою -- на неї наприкінці необхідно посилатись

В даному розділі будуть основні теоретичні відомості про об'єкт дослідження та огляд суміжних робіт в даній сфері.

\section{Задача класифікації: визначення, види}

Класифікація - це процес віднесення об'єкту за якимись його характеристиками до певної групи або груп з наперед визначеної множини груп. Класифікація може бути бінарною, багатокласовою, багатомітковою, ієрархічною та інші. Бінарна класифікація - це класифікація, коли кожному об'єкту обирається група з наперед визначеної множини груп в якій знаходиться рівно дві групи; багатокласова класифікація - це класифікація, коли кожному об'єкту обирається група з наперед визначеної множини груп в якій може знаходиться довільна кількість груп; багатоміткова класифікація - це класифікація, коли кожному об'єкту можна поставити у відповідність одразу декілька класів; ієрархічна класифікація - це класифікація, в якій класи організовані у вигляді ієрархічної структури, наприклад ми можемо класифікувати тварину спочатку за видом, потім за родом, потім за сімейством. 

Задача класифікації може зустрітися в дуже багатьох сферах, наприклад: медицина (діагностика раку на основі зображень МРТ), фінанси (класифікація позичальників як \glqq надійних\grqq\ чи \glqq ризикованих\grqq\ на основі їхньої кредитної історії), роздрібна торгівля (класифікація покупців за типами покупок для надання персоналізованих знижок), транспорт (розрізнення між легковими авто, вантажівками та мотоциклами на дорозі), освіта (ідентифікація студентів, яким потрібна додаткова допомога в певних предметах), безпека (класифікація електронних листів як \glqq безпечні,\grqq\ \glqq спам\grqq\ або \glqq фішинг\grqq), біотехнології (розпізнавання мутацій, що спричиняють хвороби). 

В поточній роботі ми зосередимося на бінарній та багатокласовій класифікації.

\section{Способи вирішення задачі класифікації}

Існує декілька способів для вирішення задачі класифікації: класичні алгоритми машинного навчання (наприклад логістична регресія~\cite{ct})

\section{(Назва третього підрозділу)}


Надамо деякі рекомендації щодо використання даного стильового файлу.

\begin{theorem}
Використовуйте оточення \emph{theorem} для теорем.
\end{theorem}
\begin{proof}
Для доведень використовуйте оточення \emph{proof}.
\end{proof}
\begin{theorem}
Нумерація відбувається автоматично
\end{theorem}
\begin{claim}
Використовуйте оточення \emph{claim} для тверджень.
\end{claim}
\begin{lemma}
Використовуйте оточення \emph{lemma} для лем.
\end{lemma}
\begin{corollary}
Використовуйте оточення \emph{corollary} для наслідків.
\end{corollary}
\begin{definition}
Використовуйте оточення \emph{definition} для визначень.
\end{definition}
\begin{example}
Використовуйте оточення \emph{example} для прикладів, на які є посилання.
\end{example}
\begin{remark}
Використовуйте оточення \emph{remark} для зауважень. Зверніть увагу, як 
веде себе команда \textbf{emph}
\end{remark}


\chapconclude{\ref{chap:review}}

Наприкінці кожного розділу ви повинні навести коротенькі підсумки по його 
результатах. Зокрема, для оглядового розділу в якості висновків необхідно 
зазначити, які задачі у даній тематиці вже були розв'язані, а саме 
поставлена вами задача розв'язана не була (або розв'язана погано), тому у 
наступних розділах ви її й розв'язуєте.

Якщо ваш звіт складається з одного розділу, пропускайте висновок до 
нього~-- він повністю включається в загальні висновки до роботи
%!TEX root = ../thesis.tex
% створюємо розділ
\chapter{Підготовка до проведення дослідження}
\label{chap:theory}

В даному розділі знаходиться огляд основних інструментів та методів аналізу та попередньої обробки даних, також ми зазначимо використані інструменти та ресурси для моделювання.

\section{Використані інструменти та ресурси}

В якості мови програмування було вибрано Python v3.11~\cite{ct18}, це ефективна та гнучка мова програмування, для розв'язання задач машинного навчання, для якої створено велику кількість бібіліотек та ресурсів, які дозволяють ефективно розв'язувати задачі, включаючи задачі бінарної та багатокласової класифікації табличних даних та картинок. Основними бібліотеками для створення моделей були бібліотеки Deap v1.4~\cite{ct19} та scikit-learn v1.4~\cite{ct20}. Обидві бібліотеки надають документацію, невеликі навчальні посібники та приклади для пришвидшення побудови моделей.

Бібліотека Deap --- це спеціалізована бібліотека для створення еволюційних алгоритмів. Ця бібліотека має реалізовані рішення для різних задач, таких як генетичне програмування, еволюційні стратегії, генетичні алгоритми та багато інших. Вона забезпечує зручний інтерфейс для налаштування та запуску еволюційних експериментів, надаючи широкий набір інструментів для маніпуляції популяціями, відбору, кросинговеру та мутацій. Основними елементами бібліотеки Deap є індивідуми, популяції, фітнес-функції, оператори генетичних алгоритмів, такі як, відбір, кросинговер та мутація. Ця бібліотека також дозволяє налаштовувати багато параметрів, таких як розмір популяції, кількість поколінь, ймовірності мутацій та кросинговеру, що робить її дуже гнучкою для різних задач. Вона підтримує паралельні обчислення, що значно прискорює процес еволюційного пошуку оптимальних рішень. В даному дослідженні буде використовуватись бібліотека Deap для реалізації $(1+\lambda)$-EA with GP encodings, що дозволяє досліджувати ефективність та керованість цього алгоритму в контексті задачі класифікації. Зокрема, ми будемо використовувати такі оператори, як турнірний відбір та одноточкову мутацію. Крім того, буде проведено аналіз впливу різних гіперпараметрів, таких як, значення $\lambda$ та глибина дерева, а також множини термінальних та внутрішніх вузлів, на якість розв'язків та швидкість конвергенції алгоритму.

Бібліотека scikit-learn --- це популярна бібліотека для машинного навчання, яка надає великий набір інструментів для задач класифікації, регресії, кластеризації, зниження розмірності та попередньої обробки даних. Вона забезпечує простий і уніфікований інтерфейс для побудови та оцінки моделей машинного навчання, що дозволяє швидко розробляти і тестувати різні алгоритми. Основні компоненти бібліотеки scikit-learn включають реалізовані алгоритми для класифікації, регресії, кластеризації та зниження розмірності, а також методи для попередньої обробки даних. В даному дослідженні бібліотека scikit-learn буде використовуватись для підготовки даних, вибору ознак, побудови та оцінки моделей класифікації. Зокрема, ми будемо використовувати стандартні підходи до попередньої обробки даних, такі як масштабування ознак, зниження розмірності та розділення даних на тренувальну та тестову вибірки. Побудова моделей буде здійснюватись з використанням алгоритму MLP. Результати класифікації будуть оцінюватись за допомогою метрик, таких як accuracy, precision, recall та f1-score. Це дозволить порівняти ефективність різних підходів та обрати найкращий алгоритм для задачі класифікації.

Також були використані наступні бібліотеки: pandas~\cite{ct21} -- для завантаження та попередньої обробки даних, optuna~\cite{ct22} -- для оптимізації гіперпараметрів моделей, torch~\cite{ct23} та torchvision~\cite{ct24} -- для обробки картинок та створення ембідінгів з моделей.

Проаналізувавши різноманітні сервіси, які надають доступ до даних, в якості вебресурсу з даними ми використовуємо вебсайт \href{https://www.kaggle.com/datasets}{https://www.kaggle.com/datasets}. Kaggle -- це платформа для змагань з машинного навчання, яка також надає великий каталог відкритих наборів даних для різноманітних задач, включаючи класифікацію, регресію та кластеризацію. Набори даних на Kaggle часто добре документовані та попередньо оброблені, що дозволяє швидко приступити до експериментів.

\section{Попередня обробка даних}

В даній роботі використовувалися наступніі набори даниих: 

-- Pima Indians Diabetes Database (посилання: \href{https://www.kaggle.com/datasets/uciml/pima-indians-diabetes-database/data}{https://www.kaggle.com/datasets/uciml/pima-indians-diabetes-database/data}) -- це набір даних, який часто використовується для задач класифікації в області біомедичних досліджень. Цей датасет був зібраний Національним інститутом діабету, шлункових і ниркових захворювань США. Набір даних містить інформацію про жінок з племені Піма, що проживають в Арізоні, та включає показники здоров'я, які можуть впливати на розвиток діабету. Датасет складається з 768 зразків, кожен з яких має 8 вхідних ознаки і два вихідних класи, які вказують на наявність або відсутність діабету. Всі ознаки числові, що дозволяє легко використовувати їх у машинному навчанні. Датасет складається з наступних ознак: Pregnancies -- кількість вагітностей у жінки; Glucose -- рівень глюкози у плазмі крові через 2 години після навантажувального тесту; Blood Pressure -- діастолічний артеріальний тиск; Skin Thickness -- товщина шкірної складки трицепса; Insulin -- рівень інсуліну у сироватці крові; BMI -- індекс маси тіла; Diabetes Pedigree Function -- функція родоводу діабету (враховує генетичну спадковість); Age -- вік пацієнта; цільова змінна: Outcome -- наявність діабету (0 - відсутній, 1 - наявний).
Цей датасет є добре збалансованим з точки зору наявності та відсутності діабету серед обстежених жінок, що робить його придатним для задач класифікації. Попередня обробка даних для цього датасету зазвичай включає масштабування ознак, обробку пропущених значень (якщо такі є) та розділення даних на тренувальну та тестову вибірки для подальшого навчання і оцінки моделей.








\chapconclude{\ref{chap:theory}}

Наприкінці розділу знову наводяться коротенькі підсумки.
%!TEX root = ../thesis.tex
\chapter{Попередня обробка даних, побудова моделей та оцінка методів}
\label{chap:practice}

В даному розділі описано попередню обробку даних, які використовувалися в дослідженні, та описано параметри побудованих моделей. Також описано процес навчання та підбору гіперпараметрів моделей. Вказана перевірка моделей на тестових даних, за допомогою метрик: точність, precision, recall та f1-score.

\section{(якийсь підрозділ)}

Подивіться, як нераціонально використовується простір, якщо не писати 
вступи до розділів. :)

Зазвичай третій розділ присвячено опису практичного застосування або 
експериментальної перевірки аналітичних результатів, одержаних у другому 
розділі роботи. Втім, це не обов'язкова вимога, і структура основної 
частини диплому більш суттєво залежить від характеру поставлених завдань. 
Навіть якщо у вас є певне експериментальне дослідження, але його загальний 
опис займає дві сторінки, то краще приєднайте його підроздіром у 
попередній розділ.

При описі експериментальних досліджень необхідно:

\begin{itemize}
\item наводити повний опис експериментів, які проводились, параметрів 
обчислювальних середовищ, засобів програмування тощо;
\item наводити повний перелік одержаних результатів у чисельному вигляді для їх можливої 
перевірки іншими особами;
\item представляти одержані результати у вигляді таблиць та графіків, 
зрозумілих людському оку;
\item інтерпретувати одержані результати з точки зору поставленої задачі 
та загальної проблематики ваших досліджень.
\end{itemize}

У жодному разі не потрібно вставляти у даний розділ тексти 
інструментальних програм та засобів (окрім того рідкісного випадку, коли 
саме тексти програм і є результатом проведення експериментів). За 
необхідності тексти програм наводяться у додатках.


\chapconclude{\ref{chap:practice}}

Висновки до останнього розділу є, фактично, підсумковими під усім 
дослідженням; однак вони повинні стостуватись саме того, що розглядалось у 
розділі.


% Створюємо висновки
\conclusions
%!TEX root = ../thesis.tex
% створюємо Висновки до всієї роботи
У ході даної роботи ми розглянули теоретичні відомості про задачі класифікації, а саме бінарну та багатокласову класифікацію, а також різні методи їх вирішення. Нами було розглянуто класичні статистичні методи, методи машинного навчання, методи глибинного навчання та генетичні алгоритми, зокрема алгоритм $(1+\lambda)$-EA with GP encodings, для вирішення задач класифікації. Додатково ми розглянули детально процес навчання моделей класифікації та метрики, які використовуються для оцінювання ефективності таких методів.

Було описано інструменти та ресурси, які були використанні для підготовки та проведення дослідження. А саме для реалізації описаних алгоритмів було використано мову програмування Python та бібліотеки Deap та scikit-learn, що забезпечили зручний інтерфейс для налаштування та запуску експериментів, а також інструменти для попередньої обробки даних, побудови та оцінки моделей класифікації. Для аналізу результатів ми обрали три різні набори даних, які представляють різноманітні задачі класифікації.

Також ми розглянули процес попередньої обробки даних, пошуку оптимальних гіперпараметрів та провели експерименти, які продемонстрували плюси та мінуси кожного з роглянутих методів. Було встановлено, що для задач, де важлива швидкість навчання, краще використовувати MLP with gradient descent. Якщо ж нам важливо контрольованість та інтерпретованість моделі, особливо при роботі з даними малої розмірності, кайкращим вибором буде $(1+\lambda)$-EA with GP encodings, але ця модель потребує значно більшого часу для тренування. Якщо ми хочемо збільшити контрольованість і при цьому не витрачати багато часу на тренування, то краще використати алгоритм MLP with single-point mutation.

Також хочемо окреслити напрямки можливих майбутніх досліджень в даній області: перше -- можна спробувати додати до операції мутації також операцію кросинговеру, наприклад батьки можуть окрім як мутувати, схрещуватись змінюючи місцями якісь піддерева між собою, друге -- можна зробити мутацію під час якої структура дерева не буде фіксованою і будуть змінюватись тільки значення вузлів, а структура дерева буде ініціалізуватися спочатку і після цього під час мутації буде обиратися піддерево та замінюватися на інше піддерево згенероване випадковим чином. Також варто спробувати приділити певний час для оптимізації бібліотеки Deap, або ж повністю з нуля реалізувати алгоритм $(1+\lambda)$-EA with GP encodings, що також потенційно може значно покращити час навчання цієї моделі.

% Додаємо бібліографію
% Якщо ви володієте магією bibtex-у, використовуйте її та модифікуйте файл 
% з бібліографією відповідним чином
%!TEX root = ../thesis.tex
% створюємо список використаної літератури
\begin{thebibliography}    
    \bibitem{sad} 
    asda

    \bibitem{dca}
    asd [Електронний ресурс]. --- Режим доступу: \url{dsf}.

 
\end{thebibliography}


%\bibliographystyle{ugost2008}
%\bibliography{thesis}

% Створюємо додатки (дивись у файли додатків для необхідних пояснень)
% Якщо ви маєте меншу або більшу кількість додатків, модифікуйте наступні 
% рядки відповідним чином
% Якщо ви не маєте додатків, просто закоментуйте наступні рядки
%!TEX root = ../thesis.tex
\append{Тексти програм}
\label{appendix:A}

Тексти інструментальних програм для проведення експериментальних досліджень необхідно 
виносити у додатки.

\section{Програма 1}

Зауважте, як змінилась нумерація.
%!TEX root = ../thesis.tex
\append{Великі рисунки та таблиці}
\label{appendix:B}

Якщо результати вашої роботи описуються величезними рисунками і таблицями 
(один аркуш та більше) у незліченній кількості, іх також необхідно 
виносити у додатки.

% (Зробити ще два додатки -- приклади відгуку та рецензії)


% Нарешті
\end{document}