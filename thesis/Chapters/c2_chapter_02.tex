%!TEX root = ../thesis.tex
% створюємо розділ
\chapter{Аналіз інструментів для проведення дослідження}
\label{chap:theory}

В даному розділі наведено огляд основних інструментів та методів аналізу та попередньої обробки даних, також ми зазначимо використані інструменти та ресурси для моделювання.

\section{Використані інструменти та ресурси}

Проаналізувавши різноманітні сервіси, які надають доступ до даних, в якості вебресурсу з даними ми використовуємо вебсайт \href{https://www.kaggle.com/datasets}{https://www.kaggle.com/datasets}. Kaggle -- це платформа для змагань з ML, яка також надає великий каталог відкритих наборів даних для різноманітних задач, включаючи класифікацію, регресію та кластеризацію. Набори даних на kaggle часто добре документовані та попередньо оброблені, що дозволяє швидко приступити до експериментів.

В якості мови програмування для вирішення задач ML було вибрано Python v3.11~\cite{ct18} завдяки його ефективності та гнучкості. Основними бібліотеками для створення моделей стали Deap v1.4~\cite{ct19} та scikit-learn v1.4~\cite{ct20}.

Deap --- це спеціалізована бібліотека для EA, що включає інструменти для генетичних алгоритмів, еволюційних стратегій та інших EA методів. Вона надає інтерфейс для налаштування та запуску еволюційних експериментів, дозволяє маніпулювати популяціями, здійснювати відбір, кросинговер та мутації, і підтримує паралельні обчислення. В цьому дослідженні використовується $(1+\lambda)$-EA with GP encoding, з турнірним відбором та одноточковою мутацією. Буде проведено аналіз впливу гіперпараметрів, таких як значення $\lambda$ та глибина дерева, на якість рішень та швидкість конвергенції алгоритму.

Scikit-learn --- це популярна бібліотека для ML, яка надає інструменти для класифікації, регресії, кластеризації, зниження розмірності та попередньої обробки даних. Вона забезпечує уніфікований інтерфейс для побудови та оцінки моделей ML. В даному дослідженні scikit-learn буде використовуватись для підготовки даних, вибору ознак, побудови та оцінки моделей класифікації, зокрема з використанням алгоритму MLP. Оцінка результатів класифікації здійснюватиметься за метриками accuracy, precision, recall та f1-score.

Також були використані бібліотеки pandas~\cite{ct21} для обробки даних, optuna~\cite{ct22} для оптимізації гіперпараметрів моделей, та torch~\cite{ct23} з torchvision~\cite{ct24} для обробки зображень і створення ембедінгів.

Для доступу до даних обрано вебсайт \href{https://www.kaggle.com/datasets}{https://www.kaggle.com/datasets}, що надає великий каталог відкритих наборів даних, добре документованих та попередньо оброблених, що дозволяє швидко приступити до експериментів.

\section{Опис використаних наборів даних} \label{sec:data-description}

В даній роботі використовувалися наступні набори даниих: 

-- Pima Indians Diabetes Database~\cite{ct30} -- це набір даних, який використовується для задач бінарної класифікації табличних даних в області біомедичних досліджень. Цей датасет був зібраний Національним інститутом діабету, шлункових і ниркових захворювань США. Набір даних містить інформацію про жінок з племені Піма, що проживають в Арізоні, та включає показники здоров'я, які можуть впливати на розвиток діабету. Датасет складається з 768 зразків, кожен з яких має 8 вхідних ознак і два вихідних класи, які вказують на наявність або відсутність діабету. Всі ознаки числові, що дозволяє легко використовувати їх у ML. Датасет складається з наступних ознак: Pregnancies -- кількість вагітностей у жінки; Glucose -- рівень глюкози у плазмі крові через 2 години після навантажувального тесту; Blood Pressure -- діастолічний артеріальний тиск; Skin Thickness -- товщина шкірної складки трицепса; Insulin -- рівень інсуліну у сироватці крові; BMI -- індекс маси тіла; Diabetes Pedigree Function -- функція родоводу діабету (враховує генетичну спадковість); Age -- вік пацієнта; цільова змінна: Outcome -- наявність діабету (0 - відсутній, 1 - наявний). Цей датасет має збалансовані класи.

-- Human Activity Recognition with Smartphones~\cite{ct31} -- це набір даних, який використовується для задач багатокласової класифікації табличних даних в області розпізнавання людської діяльності. Цей датасет був зібраний за допомогою вбудованих акселерометрів та гіроскопів смартфонів, що носили на поясі 30 учасників. Дані записувалися під час виконання різних фізичних активностей, включаючи ходьбу, підйом та спуск по сходах, сидіння, стояння та лежання. Датасет складається з 10 299 зразків, кожен з яких має 562 вхідні ознаки, що представляють різні статистичні та перетворені значення з сигналів акселерометра і гіроскопа, такі як середнє значення, стандартне відхилення, максимальні та мінімальні значення, а також частотні перетворення. Всі ознаки числові. Датасет складається з наступних ознак: Body Acceleration -- лінійне прискорення тіла в осях X, Y та Z; Total Acceleration -- загальне прискорення тіла в осях X, Y та Z; Body Gyroscope -- кутова швидкість тіла в осях X, Y та Z; Jerk Signals -- похідні лінійного прискорення та кутової швидкості; Magnitude of these three-dimensional signals -- величина сигналів прискорення та гіроскопа; Frequency domain features -- перетворені у частотну область сигнали за допомогою швидкого перетворення Фур'є; цільова змінна: Activity -- тип фізичної активності, виконуваної учасником (наприклад, walking, walking upstairs, walking downstairs, sitting, standing, laying). Цей датасет є збалансованим і добре підходить для задач класифікації, оскільки містить різноманітні фізичні активності.

-- Chest X-Ray Images (Pneumonia)~\cite{ct32} -- це набір даних, який використовується для задач бінарної та багатокласової класифікації зображень у медичних дослідженнях. Цей датасет містить рентгенівські знімки грудної клітки пацієнтів з пневмонією (вірусною або бактеріальною) та без неї. Дані були зібрані для сприяння розвитку моделей ML, здатних автоматично виявляти пневмонію на рентгенівських знімках. Датасет складається з 5 840 зображень, які розділені на дві категорії: Train та Test. Кожна категорія містить зображення, позначені як <<Pneumonia>> або <<Normal>>, або якщо розглядати задачу, як багатокласову класифікацію, то <<Pneumonia>> розділяється на <<Virus>> та <<Bacteria>>. Цей датасет надає великі можливості для досліджень у сфері медичної діагностики за допомогою глибинного навчання, дозволяючи розробляти моделі, які можуть автоматично ідентифікувати захворювання на основі рентгенівських знімків.

Ці три набори даних представляють різноманітні задачі класифікації, включаючи бінарну класифікацію табличних даних, багатокласову класифікацію табличних даних, бінарну класифікацію зображень та багатокласову класифікацію зображень. Це дозволяє комплексно оцінити ефективність методів і моделей у різних доменах застосування ML.

\section{Попередня обробка даних}

Як вже зазначалося в даній роботі використовується три набори даних, а саме: Pima Indians Diabetes Database, Human Activity Recognition with Smartphones та Chest X-Ray Images (Pneumonia). Для кожного набору даних була застосована своя попередня обробка. Далі ми наведемо, які методи обробки були застосовані до кожного набору.

Почнемо розгляд з Pima Indians Diabetes Database. Першим кроком попередньої обробки була заміна нульових значень, які зустрічаються у деяких ознаках (Pregnancies, Glucose, BloodPressure, SkinThickness, Insulin, BMI, DiabetesPedigreeFunction, Age) на відсутні значення (nan). Це дозволяє уникнути впливу невірних даних на подальший аналіз. Далі, для кожної ознаки, у якої були відсутні значення, було обчислено медіанні значення для груп з позитивним та негативним результатом по діабету (Outcome). Відсутні значення заповнювалися відповідно до медіанної величини для відповідної групи. 

Після цього було проведено обробку змінної Insulin для видалення викидів. Викиди визначалися за допомогою методу Interquartile Range Technique~\cite{ct33}. Наступним кроком було виявлення та видалення викидів за допомогою методу Local Outlier Factor~\cite{ct34}. Цей метод використовує локальну щільність сусідів для визначення аномалій. Після розрахунку негативного фактору аномалії для кожного зразка, було визначено порогове значення, і видалено ті зразки, які мали значення нижче цього порогу. 

Після видалення аномалій дані були розділені на ознаки та мітки. Вибірки було поділено на тренувальний та тестовий набори даних у пропорції 80:20. Потім, для тренувального та тестового наборів даних було проведено стандартизацію ознак шляхом видалення середнього значення та масштабування до одиничної дисперсії. В результаті ми отримали оброблений датасет, який будемо використовувати для порівняння моделей для задачі бінарної класифікації табличних даних.

Наступну попередню обробку опишемо для набору даних Human Activity Recognition with Smartphones. Для цього набору даних було застосовано наступні методи попередньої обробки. По-перше, дані було розділено на тренувальну та тестову вибірки, кожна з яких містила відповідні дані для тренування та тестування моделей. Після завантаження датасетів для тренування та тестування, було застосовано LabelEncoder для кодування міток активностей (Activity) у числовий формат. Наступним кроком було розділення даних на ознаки та мітки для тренувальних і тестових вибірок. Далі було проведено випадкове перемішування тренувальних та тестових даних для уникнення впливу можливого порядку даних на результати навчання моделей. Для покращення роботи моделей було проведено стандартизацію ознак шляхом видалення середнього значення та масштабування до одиничної дисперсії. Цей процес дозволяє моделі краще адаптуватися до даних, які мають різний масштаб. В результаті попередньої обробки ми отримали стандартизовані та перемішані тренувальні та тестові набори даних, готові до подальшого використання у моделюванні. 

Нарешті, розглянемо попередню обробку даних для набору даних Chest X-Ray Images (Pneumonia). Для цього набору даних було застосовано наступні методи попередньої обробки. По-перше, дані були завантажені та організовані у вигляді класів, де кожне зображення має відповідну мітку (0 -- нормальний, 1 -- бактеріальна пневмонія, 2 -- вірусна пневмонія). Як зазначалося ми використали цей датасет для бінарної та багатокласової класифікації, для бінарної, відповідно, класи бактеріальна та вірусна пневмонії були об'єднані в один клас -- пневмонія, а для багатокласової використовувалися класи нормальний, бактеріальна пневмонія та вірусна пневмонія. 

Зображення були перетворені до розміру 224x224 пікселів і конвертовані в градації сірого для уніфікації формату. Для екстракції ознак було використано попередньо натреновану модель ResNet-50~\cite{ct35} без останнього повнозв'язного шару. Модель була завантажена зі збережених ваг (які ми самі натренували використовуючи цей же набір даних) та використана для отримання векторів ознак зображень. Далі, для обробки отриманих векторів ознак, було застосовано стандартизацію ознак шляхом видалення середнього значення та масштабування до одиничної дисперсії. Для зменшення розмірності та збереження 99$\%$ варіативності даних було застосовано метод Principal Component Analysis~\cite{ct36}. В результаті ми отримали оброблені вектори ознак, готові до подальшого використання у моделюванні.

\chapconclude{\ref{chap:theory}}

У цьому розділі проаналізовано інструменти та методи обробки даних. Джерелом даних був Kaggle. Використано Python 3.11 з бібліотеками Deap 1.4 для еволюційних алгоритмів і scikit-learn 1.4 для машинного навчання.

Для набору даних Pima Indians Diabetes Database проведено заміну нульових значень, заповнення відсутніх медіанними величинами, видалення викидів і стандартизацію ознак.

Для Human Activity Recognition with Smartphones виконано кодування міток активностей, перемішування даних і стандартизацію ознак.

Для Chest X-Ray Images (Pneumonia) змінено розмір зображень, конвертовано в градації сірого, використано модель ResNet-50 для екстракції ознак, стандартизацію та зменшення розмірності.

Додамо, що усі методи тренувалися на комп'ютері, який має наступні обчислювальні ресурси: процесор -- AMD Ryzen 5 3600 6-Core Processor, оперативна пам'ять -- 32GB DDR4.