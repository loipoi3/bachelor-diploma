%!TEX root = ../abstract.tex

\abstractUkr

Кваліфікаційна робота містить: 55 стор., 6 рисунки, 20 таблиць, 37 джерел.

У даній роботі розглядаються методи для вирішення задач класифікації, а саме: MLP, який використовує оптимізаційні алгоритми в основі яких градієнтний спуск, MLP, який використовує оптимізаційний алгоритм на основі одноточкової мутації та $(1+\lambda)$-EA with GP encoding. Ці методи порівнювались на задачах бінарної та мультикласової класифікації табличних даних та картинок.

У ході дослідження було встановлено, що всі три методи здатні досягти однакових метрик у всіх задачах. Найшвидшу збіжність до цих метрик продемонстрував MLP з використанням градієнтного спуску. Тим не менш, $(1+\lambda)$-EA with GP encoding виділився завдяки здатності легко адаптуватись до умов задачі. Цей метод дозволяє обирати кількість нащадків і регулювати експресивність індивідів, що надає можливість зосередити пошук рішень у конкретних областях простору рішень. Такий підхід є особливо корисним, коли потрібно зосередитися на важливих регіонах пошуку для вдосконалення рішень.


% наприкінці анотації потрібно зазначити ключові слова
\MakeUppercase{МАШИННЕ НАВЧАННЯ, ЕВОЛЮЦІЙНІ АЛГОРИТМИ, ГЕНЕТИЧНЕ ПРОГРАМУВАННЯ, МЕТОДИ ОПТИМІЗАЦІЇ, ЕКСПРЕСИВНІ КОДУВАННЯ}


%%%% Рішенням кафедри з 2018 року ми прибираємо анотації російською мовою
% \abstractRus
%
%Русская аннотация должна быть точным переводом украинской (включая 
%статистику и ключевые слова).

\abstractEng

This paper considers methods for solving classification problems, namely: MLP, which uses optimization algorithms based on gradient descent, MLP, which uses an optimization algorithm based on one-point mutation, and $(1+\lambda)$-EA with GP encoding. These methods were compared in the tasks of binary and multiclass classification of tabular data and pictures.

During the research, it was established that all three methods are able to achieve the same metrics in all tasks. The fastest convergence to these metrics was demonstrated by MLP using gradient descent. Nevertheless, $(1+\lambda)$-EA with GP encoding stood out due to its ability to easily adapt to the task conditions. This method allows you to choose the number of offsprings and regulate the expressiveness of individuals, which makes it possible to focus the search for solutions in specific areas of the solution space. This approach is particularly useful when focusing on important search regions to improve solutions.

% наприкінці анотації потрібно зазначити ключові слова
\MakeUppercase{MACHINE LEARNING, EVOLUTIONARY ALGORITHMS, GENETIC PROGRAMMING, OPTIMIZATION METHODS, EXPRESSIVE ENCODINGS}

% Не прибирайте даний рядок
\clearpage