%!TEX root = ../thesis.tex
% створюємо вступ
\textbf{Актуальність дослідження.} Наразі існує дуже багато методів, які вирішують задачу класифікації: статистичні методи, методи на основі нейронних мереж, методи класичного машинного навчання. Але у цих методів є свої недоліки: методи на основі нейронних мереж, мають дуже погану інтерпретабельність та контрольованість, методи класичного машинного навчання та статистичні методи мають непогану інтерпретабельність (в залежності від методу), але також не найкращу контрольваність. Задача класифікації відіграє дуже важливу роль у сучасній науці, для прикладу, іноді потрібно класифікувати чи у людини є якась хвороба легень, маючи рентгенівський знімок легень, або класифікувати чи має людина якусь хворобу, спираючись на різні її показники, як от вага, рівень цукру в крові та інші. Можливість контролювати роботу алгоритму відіграє в цьому завданні важливу роль, оскільки це дозволяє нам налаштувати алгоритм під наші потреби і при цьому ми будемо точно знати чому та як він працює.

\textbf{Метою дослідження} є пошук найкращого методу класифікації, на прикладі методів MLP with gradient descent, MLP with single-point mutation, $(1+\lambda)$-EA with GP encodings, для підвищення продуктивності у розв'язанні задач бінарної та мультикласової класифікації табличних даних та картинок.

\emph{Об'єктом дослідження} є методи класифікації даних для машинного навчання на прикладі задачі бінарної та мультикласової класифікації.

\emph{Предметом дослідження} є особливості контролювання алгоритмів на прикладі MLP with gradient descent, MLP with single-point mutation, $(1+\lambda)$-EA with GP encodings на прикладі застосування до задач бінарної та мультикласової класифікації табличних даних та картинок.

\textbf{Наукова новизна} полягає в дослідженні та порівнянні алгоритмів MLP with gradient descent, MLP with single-point mutation, $(1+\lambda)$-EA with GP encodings на прикладі задач бінарної та мультикласової класифікації.

\textbf{Практичне значення} результатів полягає в використанні перелічених вище методів, для задачі класифікації, для покращення контрольованості і збереженню такої ж точності та швидкості, як і в класичних методах.