%!TEX root = ../thesis.tex
% створюємо вступ
\textbf{Актуальність дослідження.} Використання класифікаційних задач має широкий спектр застосування в різних сферах наукових досліджень та практичних областях. Наприклад, важливо класифікувати, чи особа є носієм певного захворювання, спираючись на ретгенівські знімки або аналізи крові, що дозволяє з високою точністю визначати наявність патологій. З цього випливає необхідність дослідження алгоритмів, які вирішують задачу класифікації, для того, щоб мати більшу гнучкість у налаштуванні процесу пошуку по певним областям простору рішень. Відповідно до цього актуальність дослідження полягає у порівнянні алгоритимів MLP та $(1+\lambda)$-EA with GP encodings, щоб перевірити чи надає останній можливість контролювати гіперпараметри для більш детального пошуку та можливість краще адаптуватися до поточної задачі.

\textbf{Метою дослідження} є пошук оптимального методу класифікації серед методів MLP with gradient descent, MLP with single-point mutation, $(1+\lambda)$-EA with GP encodings, для дослідження контрольованості (див. означення у розділі перелік умовних позначень, скорочень і термінів) у розв'язанні задач бінарної та мультикласової класифікації табличних даних та картинок.

\emph{Об'єктом дослідження} є якісна поведінка MLP та $(1+\lambda)$-еволюційних алгоритмів для задачі бінарної та мультикласової класифікації.

\emph{Предметом дослідження} є особливості контролювання алгоритмів на прикладі MLP with gradient descent, MLP with single-point mutation, $(1+\lambda)$-EA with GP encodings на прикладі застосування до задач бінарної та мультикласової класифікації табличних даних та картинок.

\textbf{Наукова новизна} полягає в дослідженні та порівнянні алгоритмів MLP with gradient descent, MLP with single-point mutation, $(1+\lambda)$-EA with GP encodings на прикладі задач бінарної та мультикласової класифікації.

\textbf{Практичне значення} результатів полягає в використанні перелічених вище методів, для задачі класифікації, для покращення контрольованості і збереженню такої ж точності та швидкості, як і в класичних методах.