%!TEX root = ../thesis.tex
% створюємо вступ
\textbf{Актуальність дослідження.} Використання класифікаційних задач має широкий спектр застосування в різних сферах наукових досліджень та практичних областях. Наприклад, важливо класифікувати, чи особа є носієм певного захворювання, спираючись на рентгенівські знімки або аналізи крові. Алгоритми, які вирішують задачі класифікації, дозволяють робити дешеві та швидкі попередні аналізи для того, щоб дізнатися чи є людина хворою. Існують різні підходи до вирішення задач класифікації, але вони часто мають низьку контрольованість. Тому є потреба у досліджені алгоритмів, щоб перевірити, чи надають вони більшу гнучкість в контролюванні гіперпараметрів для більш детального пошуку по певним областям простору рішень та можливість краще адаптуватися до поточної задачі.

\textbf{Метою дослідження} є пошук оптимального методу вирішення задач бінарної та багатокласової класифікації табличних даних та зображень серед методів MLP with gradient descent, MLP with single-point mutation, $(1+\lambda)$-EA with GP encoding.

\emph{Об'єктом дослідження} є якісна поведінка MLP та $(1+\lambda)$-еволюційних алгоритмів для задачі бінарної та мультикласової класифікації.

\emph{Предметом дослідження} є методи вирішення задач бінарної та багатокласової класифікації, зокрема точність, швидкість та збіжність цих методів для табличних даних та зображень на основі алгоритмів MLP with gradient descent, MLP with single-point mutation та $(1+\lambda)$-EA with GP encoding.

\textbf{Наукова новизна} полягає в дослідженні та порівнянні алгоритмів MLP with gradient descent, MLP with single-point mutation, $(1+\lambda)$-EA with GP encoding на прикладі задач бінарної та мультикласової класифікації.

\textbf{Практичне значення} результатів полягає в використанні перелічених вище методів, для задачі класифікації, для покращення контрольованості і збереженню такої ж точності та швидкості, як і в класичних методах.