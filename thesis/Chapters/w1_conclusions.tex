%!TEX root = ../thesis.tex
% створюємо Висновки до всієї роботи
Загальні висновки до роботи повинні підсумовувати усі ваші досягнення у 
даному напрямку досліджень.

За кожним пунктом завдань, поставлених у вступі, у висновках повинен 
міститись звіт про виконання: виконано, не виконано, виконано частково (І 
чому саме так). Наприклад, якщо першим поставленим завданням у вас іде 
<<огляд літератури за тематикою досліджень>>, то на початку висновків ви 
повинні зазначити, що <<у ході даної роботи був проведений аналіз 
опублікованих джерел за тематикою (...), який показав, що (...)>>. Окрім 
простої констатації про виконання ви повинні навести, які саме результати 
ви одержали та проінтерпретувати їх з точки зору поставленої задачі, мети 
та загальної проблематики.

В ідеалі загальні висновки повинні збиратись з висновків до кожного 
розділу, але ідеал недосяжний. :) Однак висновки не повинні містити 
формул, таблиць та рисунків. Дозволяється (та навіть вітається) 
використовувати числа (на кшталт <<розроблена методика дозволяє підвищити 
ефективність пустопорожньої балаканини на $2.71\%$>>).

Наприкінці висновків необхідно зазначити напрямки подальших досліджень: 
куди саме, як вам вважається, необхідно прямувати наступним дослідникам у 
даній тематиці.
