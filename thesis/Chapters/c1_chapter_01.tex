%!TEX root = ../thesis.tex

\chapter{Методи та підходи вирішення задач класифікації}
\label{chap:review}  %% відмічайте кожен розділ певною міткою -- на неї наприкінці необхідно посилатись

В даному розділі будуть основні теоретичні відомості про об'єкт дослідження та огляд суміжних робіт в даній сфері.

\section{Задача класифікації: визначення, види}

Класифікація - це процес віднесення об'єкту за якимись його характеристиками до певної групи або груп з наперед визначеної множини груп. Класифікація може бути бінарною, багатокласовою, багатомітковою, ієрархічною та інші. Бінарна класифікація - це класифікація, коли кожному об'єкту обирається група з наперед визначеної множини груп в якій знаходиться рівно дві групи; багатокласова класифікація - це класифікація, коли кожному об'єкту обирається група з наперед визначеної множини груп в якій може знаходиться довільна кількість груп; багатоміткова класифікація - це класифікація, коли кожному об'єкту можна поставити у відповідність одразу декілька класів; ієрархічна класифікація - це класифікація, в якій класи організовані у вигляді ієрархічної структури, наприклад ми можемо класифікувати тварину спочатку за видом, потім за родом, потім за сімейством. 

Задача класифікації може зустрітися в дуже багатьох сферах, наприклад: медицина (діагностика раку на основі зображень МРТ), фінанси (класифікація позичальників як \glqq надійних\grqq\ чи \glqq ризикованих\grqq\ на основі їхньої кредитної історії), роздрібна торгівля (класифікація покупців за типами покупок для надання персоналізованих знижок), транспорт (розрізнення між легковими авто, вантажівками та мотоциклами на дорозі), освіта (ідентифікація студентів, яким потрібна додаткова допомога в певних предметах), безпека (класифікація електронних листів як \glqq безпечні,\grqq\ \glqq спам\grqq\ або \glqq фішинг\grqq), біотехнології (розпізнавання мутацій, що спричиняють хвороби). 

В поточній роботі ми зосередимося на бінарній та багатокласовій класифікації.

\section{Способи вирішення задачі класифікації}

Існує декілька способів для вирішення задачі класифікації: класичні алгоритми машинного навчання (наприклад логістична регресія~\cite{ct})

\section{(Назва третього підрозділу)}


Надамо деякі рекомендації щодо використання даного стильового файлу.

\begin{theorem}
Використовуйте оточення \emph{theorem} для теорем.
\end{theorem}
\begin{proof}
Для доведень використовуйте оточення \emph{proof}.
\end{proof}
\begin{theorem}
Нумерація відбувається автоматично
\end{theorem}
\begin{claim}
Використовуйте оточення \emph{claim} для тверджень.
\end{claim}
\begin{lemma}
Використовуйте оточення \emph{lemma} для лем.
\end{lemma}
\begin{corollary}
Використовуйте оточення \emph{corollary} для наслідків.
\end{corollary}
\begin{definition}
Використовуйте оточення \emph{definition} для визначень.
\end{definition}
\begin{example}
Використовуйте оточення \emph{example} для прикладів, на які є посилання.
\end{example}
\begin{remark}
Використовуйте оточення \emph{remark} для зауважень. Зверніть увагу, як 
веде себе команда \textbf{emph}
\end{remark}


\chapconclude{\ref{chap:review}}

Наприкінці кожного розділу ви повинні навести коротенькі підсумки по його 
результатах. Зокрема, для оглядового розділу в якості висновків необхідно 
зазначити, які задачі у даній тематиці вже були розв'язані, а саме 
поставлена вами задача розв'язана не була (або розв'язана погано), тому у 
наступних розділах ви її й розв'язуєте.

Якщо ваш звіт складається з одного розділу, пропускайте висновок до 
нього~-- він повністю включається в загальні висновки до роботи