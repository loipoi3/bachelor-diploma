%!TEX root = ../thesis.tex
% створюємо перелік умовних позначень, скорочень і термінів
ML --- машинне навчання (англ. Machine Learning)

MLP --- багатошаровий перцептрон (англ. Multilayer Perceptron)

EA --- еволюційний алгоритм (англ. Evolutionary Algorithm)

GP --- генетичне програмування (англ. Genetic Programming)

Adam --- адаптивна оцінка моменту (англ. Adaptive Moment Estimation)

$(1+\lambda)$-EA with GP encodings --- еволюційний алгоритм, який може використовуватися для вирішення задач класифікації (англ. $(1+\lambda)$-Evolutionary Algorithm with Genetic Programming encodings).

MLP with gradient descent --- багатошаровий перцептрон, який використовує метод на основі градієнтного спуску, в якості оптимізаційного алгоритму.

MLP with single-point mutation --- багатошаровий перцептрон, який використовує одноточкову мутацію, в якості оптимізаційного алгоритму.

ReLU --- випрямлений лінійний вузол (англ. Rectified Linear Unit).

Індивід --- це об'єкт, який містить в собі всю необхідну інформації для потенційного вирішення задачі.

Популяція --- це набір індивідів.

Фітнес-функція --- функція, яка вимірює наскільки добре пристосований кожен індивід в популяції, до зовнішньої середи.

Кросинговер --- операція в якій генерується новий індивід шляхом, якогось комбінування двох індивідів з минулого покоління (батьків).

Мутація --- якась модифікація індивіду.

Контрольованість у контексті $(1+\lambda)$-еволюційного алгоритму з генетичним програмуванням --- визначається як здатність алгоритму дозволяти користувачу точно регулювати його параметри (наприклад, кількість нащадків $\lambda$ і стратегії мутації), щоб оптимізувати процес пошуку рішення та адаптувати його під специфічні умови задачі.