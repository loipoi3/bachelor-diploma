%!TEX root = ../thesis.tex
% створюємо перелік умовних позначень, скорочень і термінів
ML --- машинне навчання (англ. Machine Learning)

MLP --- багатошаровий перцептрон (англ. Multilayer Perceptron)

EA --- еволюційний алгоритм (англ. Evolutionary Algorithm)

GP --- генетичне програмування (англ. Genetic Programming)

Adam --- адаптивна оцінка моменту (англ. Adaptive Moment Estimation)

$(1+\lambda)$-EA with GP encodings --- еволюційний алгоритм, який може використовуватися для вирішення задач класифікації (англ. $(1+\lambda)$-Evolutionary Algorithm with Genetic Programming encodings).

MLP with gradient descent --- багатошаровий перцептрон, який використовує метод на основі градієнтного спуску, в якості оптимізаційного алгоритму.

MLP with single-point mutation --- багатошаровий перцептрон, який використовує одноточкову мутацію, в якості оптимізаційного алгоритму.

Фітнес-функція --- це функція \( F: \mathcal{S} \rightarrow \mathbb{R} \), яка ставить у відповідність представлення рішення \( S \) на дійсне число \( f \).

Індивід --- \( I \) в еволюційних алгоритмах визначається як пара \( I = (S, f) \), де \( S \) є представленням рішення в просторі рішень \( \mathcal{S} \), природа \( S \) залежить від конкретної проблеми та може варіюватися в широких межах, від двійкових рядків, дійсних векторів, дерев до більш складних структур даних, \( f \) — це значення фітнес-функції, пов’язане з індивідом, кількісно оцінюючи якість індивіда як рішення цільової проблеми.

Популяція --- \( P \) визначається як множина індивідів \( P = \{I_1, I_2, \ldots, I_N\} \), де кожен окремий \( I_i \) є варіантом вирішення розв'язуваної проблеми.

Кросинговер --- \( C \), є бінарною функцією, яка бере два індивіди з популяції як вхідні дані та створює одного або більше нащадків, потенційно включаючи генетичний матеріал від обох батьків. Формально це можна виразити так: \( C: (I_i, I_j) \rightarrow (I_{i'}, I_{j'}) \), де \( I_i \) і \( I_j \) є батьківськими індивідами, кожен з яких містить представлення рішення та значення фітнес-функції, \( I_{i'} \) і \( I_{j'} \) є індивідами-нащадками, отриманими в результаті операції кросинговеру.

Мутація --- це функція \( M: I \rightarrow I' \), де \( I \) — оригінальний індивід, \( I' \) є мутованим індивідом з потенційно зміненим представленням рішення \( S' \) і відповідним новим значенням фітнес-функції \( f' \), яка застосовує стохастичну модифікацію до індивіду, що потенційно призводить до появи нового варіанту рішення.

Контрольованість у контексті $(1+\lambda)$-еволюційного алгоритму з генетичним програмуванням --- визначається як здатність алгоритму дозволяти користувачу точно регулювати його параметри (наприклад, кількість нащадків $\lambda$ і стратегії мутації), щоб оптимізувати процес пошуку рішення та адаптувати його під специфічні умови задачі.

Precision --- це метрика, яка визначає відношення кількості правильно класифікованих позитивних прикладів до загальної кількості прикладів, що були класифіковані як позитивні.

Recall --- це метрика, яка визначає відношення кількості правильно класифікованих позитивних прикладів до загальної кількості справді позитивних прикладів.

F1-score --- це гармонійне середнє між precision та recall.

Expressive Encodings --- експресивне кодування.